\documentclass[12pt]{article}
\usepackage[english]{babel}
\usepackage{blindtext}
\usepackage{pgfplots}
\usepackage[a4paper, inner=1.5cm, outer=3cm, top=2cm,bottom=3cm, bindingoffset=1cm]{geometry}
\begin{document}
\begin{titlepage}
    \begin{center}
        \vspace*{1cm} 
        \Huge
        \textbf{Machine Learning in Healthcare} 
        \vspace{0.5cm}
        \normalsize
        \vspace{0cm}
        Analysis of Machine Learning Algortihms with the Pima Indians Diabetes dataset.
 
        \vspace{1.5cm}
 
        \textbf{Alexander Roque Rodrigues}
 
        \vfill
 
        A dissertation presented for the degree of\\
        Bachelor in Computer Science
 
        \vspace{0.8cm}
  
        \Large
        Department of Computer Science\\        
        Smt. Parvatibai Chowgule College of Arts and Science\\
        India\\
        18 August 2019
 
    \end{center}
\end{titlepage}
\Huge
\newpage
\huge
\textbf{Acknowledgements}
\normalsize
\newpage
\tableofcontents
\newpage
\part{Review of Literature}
\newpage
\part{Background}
\newpage
\part{Machine Learning Algorithms}
\newpage
\section{Linear Regression}
\subsection{Introduction to Linear Regression}
Linear regression is a forecasting technique that can be use to predict the future of a number series based on the historic data given. Linear Regression is the go to choice for many predictions since the linear regression algorithm.

\begin{itemize}
  \item produces decent and  easy to interpret results.
  \item is computationally inexpensive.
  \item conversion of algorithm into code does not take much effort or time.
  \item numeric values as well as nominal values support is offered.
\end{itemize}

However, a major drawback of linear regression is that it \textbf{poorly models nonlinear data}.
\subsection{Deriving the Algorithm}

\begin{equation}
y = \beta_{0}+\beta_{1}+\epsilon
\end{equation}
The most basic equation for linear regression can be expressed via this simple equation.
\subsection{Potential Areas of Use}
\newpage
\section{Logistic Regression}
\newpage
\section{K-Nearest Neighbours}
\newpage
\section{Decision Tree}
\newpage
\section{Random Forest}
\newpage
\section{Gradient Boosting}
\newpage
\section{Support Vector Machine}
\newpage
\section{Perceptron}
\newpage
\section{Multilayered Perceptron}
\newpage
\part{Implementing the Algorithms}
\newpage
\part{Future Scope}
\newpage
\part{Conclusions}
\clearpage

\newpage
\begin{thebibliography}{}

\bibitem{WMG} 
Wei M, Gibbons LW, Mitchell TL \textit{et al}. (1999) The Association between cardiorespiratory fitness and impaired fasting glucose and type 2 diabetes mellitus in men. \textit{Ann Intern Med} \textbf{130}, 427-34.

\end{thebibliography}
\end{document}
